\section{Envolvent injectiu sobre cossos residuals}
Recordem que un anell $R$ és local si només té un ideal maximal $\mathfrak{m}$. En aquest cas, diem que $R/\mathfrak{m}$ és el cos residual de $R$.
\subsection{Resultats sobre anells artinians locals}
Sigui $M$ $R$-mòdul ($R$ no necessàriament local). Definim la longitud de $M$ com
\begin{equation*}
    \ell_{R}(M):=\sup\{n\in\mathbb{N}:\exists(0\subsetneq M_{1}\subsetneq\ldots\subsetneq M_{n}=M)\}
\end{equation*}
Es pot comprovar que $\ell_{R}$ és una funció additiva, és a dir, donada una successió exacta curta a $\textrm{Mod}_{R}$
\begin{center}
    \begin{tikzcd}
    0
    \arrow{r}
    &A
    \arrow{r}
    &B
    \arrow{r}
    &C
    \arrow{r}
    &0
\end{tikzcd}
\end{center}
tenim que $\ell_{R}(B)=\ell_{R}(A)+\ell_{R}(C)$.\newline
Considerarem en aquesta secció anells artinians locals. D'aquestes condicions, es dedueix que l'anell artinià local és noetherià i té $R$ té longitud finita com a $R$-mòdul.
\begin{lema}\label{artinian-local-lenght}
    Sigui $(R,\mathfrak{m})$ anell artinià local, $M$ $R$-mòdul finitament generat. Aleshores,
    \begin{equation*}
        \ell_{R}(M)=\ell_{R}(\Hom_{R}(M,E_{R}(R/\mathfrak{m})))
    \end{equation*}
    \begin{proof}
        Tenim que
        \begin{align*}
            \Hom_{R}(R/\mathfrak{m},E_{R}(R/\mathfrak{m}))
            &\cong E_{R/\mathfrak{m}}(R/\mathfrak{m})
            &\quad&\textrm{(Isomorfisme via $\varphi\mapsto\varphi(1_{R}+\mathfrak{m})$)}\\
            &\cong R/\mathfrak{m}
            &\quad&\textrm{(Tot mòdul sobre un cos és injectiu i $R/\mathfrak{m}$ és minimal)}
        \end{align*}
        Aleshores, 
        \begin{equation*}
            \ell_{R}(R/\mathfrak{m})=\ell_{R}(\Hom_{R}(M,E_{R}(R/\mathfrak{m})))
        \end{equation*}
        Procedim per inducció en $\ell_{R}(M)$, $M$ $R$-mòdul finitament generat. Tenim que existeix $\pi:M\twoheadrightarrow R/\mathfrak{m}$ morfisme de $R$-mòduls exhaustiu. Considerem la successió exacta curta
        \begin{center}
        \begin{tikzcd}
            0
            \arrow{r}
            &\ker{\pi}
            \arrow{r}{\iota}
            &M
            \arrow{r}{\pi}
            &R/\mathfrak{m}
            \arrow{r}
            &0
        \end{tikzcd}
        \end{center}
        Com $E_{R}(R/\mathfrak{m})$ és injectiu, obtenim la successió exacta curta
        \begin{center}
        \begin{tikzcd}
            0
            \arrow{r}
            &\Hom_{R}(R/\mathfrak{m},E_{R}(R/\mathfrak{m}))
            \arrow{r}{\pi_{*}}
            &\Hom_{R}(M,E_{R}(R/\mathfrak{m}))
            \arrow{r}{\iota_{*}}
            &\Hom_{R}(\ker{\pi},E_{R}(R/\mathfrak{m}))
            \arrow{r}
            &0
        \end{tikzcd}
        \end{center}
        De les darreres successions exactes curtes, deduïm que
        \begin{align*}
            &\ell_{R}(\Hom_{R}(M,E_{R}(R/\mathfrak{m})))\\
            =\,
            &\ell_{R}(\Hom_{R}(R/\mathfrak{m},E_{R}(R/\mathfrak{m})))
            +\Hom_{R}(\ker{\pi},E_{R}(R/\mathfrak{m}))
            &\quad&\textrm{(Per additivitat de $\ell_{R}$)}\\
            =\,
            &\ell_{R}(R/\mathfrak{m})
            +\ell_{R}(\ker{\pi})
            &\quad&\textrm{(Hipòtesi d'inducció)}\\
            =\,
            &\ell_{R}(M)
            &\quad&\textrm{(Per additivitat de $\ell_{R}$)}
        \end{align*}
        com volíem veure.
    \end{proof}
\end{lema}
\begin{corolari}\label{artinian-local-isomorphism}
    Sigui $(R,\mathfrak{m})$ anell artinià local, $M$ $R$-mòdul finitament generat. Aleshores,
    \begin{equation*}
        M\cong\Hom_{R}(\Hom_{R}(M,E_{R}(R/\mathfrak{m})),E_{R}(R/\mathfrak{m}))
    \end{equation*}
    En particular, $R\cong\Hom_{R}(E_{R}(R/\mathfrak{m}),E_{R}(R/\mathfrak{m}))$.
    \begin{proof}
        Per tot $m\in M-\{0\}$, existeix un morfisme de $R$-mòduls no trivial $h_{m}:(m)\rightarrow R/\mathfrak{m}$. Com $E_{R}(R/\mathfrak{m})$ injectiu i $R/\mathfrak{m}\subset E_{R}(R/\mathfrak{m})$,
        \begin{center}
        \begin{tikzcd}
            &&E_{R}(R/\mathfrak{m})
            &&\\\\
            &&R/\mathfrak{m}
            \arrow[hook]{uu}{\iota'}
            &&\\\\
            0
            \arrow{rr}
            &&(m)
            \arrow[hook]{rr}{\iota}
            \arrow{uu}{h_{m}}
            &&M
            \arrow[swap,dotted]{uuuull}{\exists(\iota'\circ h_{m})'}
        \end{tikzcd}
        \end{center}
        Per tant, el morfisme $M\rightarrow\Hom_{R}(\Hom_{R}(M,E_{R}(R/\mathfrak{m})),E_{R}(R/\mathfrak{m}))$ definit via $m\mapsto((\iota'\circ h_{m})'\mapsto(\iota'\circ h_{m})'(m))$ esta ben definit i és injectiu (ja que per construcció té nucli trivial). A més, com
        \begin{align*}
            \ell_{R}(\Hom_{R}(\Hom_{R}(M,E_{R}(R/\mathfrak{m})),E_{R}(R/\mathfrak{m})))
            &=\ell_{R}(\Hom_{R}(M,E_{R}(R/\mathfrak{m})))
            &\quad&\textrm{(Per \ref{artinian-local-lenght})}\\
            &=\ell_{R}(M)
            &\quad&\textrm{(Per \ref{artinian-local-lenght})}
        \end{align*}
        deduïm que és isomorfisme. El darrer isomorfisme resulta de
        \begin{align*}
            R
            &\cong\Hom_{R}(\Hom_{R}(R,E_{R}(R/\mathfrak{m})),E_{R}(R/\mathfrak{m}))\\
            &\cong
            \Hom_{R}(E_{R}(R/\mathfrak{m}),E_{R}(R/\mathfrak{m}))
            &\quad&\textrm{($E_{R}(R/\mathfrak{m})\cong\Hom_{R}(R,E_{R}(R/\mathfrak{m}))$)}
        \end{align*}
        on el darrer isomorfisme és general per tot $R$-mòdul.
    \end{proof}
\end{corolari}
\subsection{Resultats sobre anells noetherians locals}
Volem aplicar \ref{env-inj-aniquillador} al ideal $\mathfrak{m}^{n}$.
\begin{proposicio}
    Sigui $(R,\mathfrak{m})$ un anell noetherià local. Aleshores,
    \begin{enumerate}
        \item $E_{R/\mathfrak{m}^{n}}(R/\mathfrak{m})\cong(0:_{E_{R}(R/\mathfrak{m})}\mathfrak{m}^{n})$.
        \begin{proof}
            $\mathfrak{m}^{n}$ és un ideal de $R$ i $\mathfrak{m}^{n}(R/\mathfrak{m})=\{0\}$. Per \ref{env-inj-aniquillador}, $E_{R/\mathfrak{m}^{n}}(R/\mathfrak{m})\cong(0:_{E_{R}(R/\mathfrak{m})}\mathfrak{m}^{n})$.
        \end{proof}
        \item $E_{R}(R/\mathfrak{m})=\bigcup_{n>0}(0:_{E_{R}(R/\mathfrak{m})}\mathfrak{m}^{n})$.
        \begin{proof}
        Com $R$ és noetherià i $E_{R}(R/\mathfrak{m})$ és un $R$-mòdul injectiu, $\bigcup_{n>0}(0:_{E_{R}(R/\mathfrak{m})}\mathfrak{m}^{n})$ és un $R$-submòdul injectiu d'$E_{R}(R/\mathfrak{m})$ per \ref{lema-tecnic-noetheria-injectiu}. Fixem-nos que
        \begin{equation*}
            R/\mathfrak{m}\subset(0:_{E_{R}(R/\mathfrak{m})}\mathfrak{m}^{n})\subset E_{R}(R/\mathfrak{m})
        \end{equation*}
        Com $E_{R}(R/\mathfrak{m})$ és l'envolvent injectiu de $R/\mathfrak{m}$, $E_{R}(R/\mathfrak{m})$ és l'injectiu més petit que conté $R/\mathfrak{m}$, d'on resulta $E_{R}(R/\mathfrak{m})\subset\bigcup_{n>0}(0:_{E_{R}(R/\mathfrak{m})}\mathfrak{m}^{n})$ i, per tant, $E_{R}(R/\mathfrak{m})=\bigcup_{n>0}(0:_{E_{R}(R/\mathfrak{m})}\mathfrak{m}^{n})$.
    \end{proof}
    \end{enumerate}
\end{proposicio}
\begin{corolari}\label{lim-inj-envolvent-cos-residual}
    Sigui $(R,\mathfrak{m})$ anell noetherià local. Aleshores, $\varinjlim E_{R/\mathfrak{m}^{n}}(R/\mathfrak{m})\cong E_{R}(R/\mathfrak{m})$.
\end{corolari}
En particular, tenim la filtració $\{E_{R/\mathfrak{m}^{n}}(R/\mathfrak{m}):n\in\mathbb{N}\}$. Els endomorfismes preserven la filtració, ja que
\begin{equation*}
    \im\Big(\varphi:(0:_{E_{R}(R/\mathfrak{m})}\mathfrak{m}^{n})\rightarrow\bigcup_{n>0}(0:_{E_{R}(R/\mathfrak{m})}\mathfrak{m}^{n})\Big)
    \subset(0:_{E_{R}(R/\mathfrak{m})}\mathfrak{m}^{n})
\end{equation*}
En efecte, de forma heurística, tenim que $f((0:_{E_{R}(R/\mathfrak{m})}\mathfrak{m}^{n}))\mathfrak{m}^{n}=f((0:_{E_{R}(R/\mathfrak{m})}\mathfrak{m}^{n})\mathfrak{m}^{n})=f(\{0\})=\{0\}$, d'on es dedueix la inclusió. Aquesta preservació simplifica la demostració del resultat següent.
\begin{teorema}\label{m-adic-isomorphism}
    Sigui $(R,\mathfrak{m})$ anell noetherià local. Aleshores, $\Hom_{R}(E_{R}(R/\mathfrak{m}),E_{R}(R/\mathfrak{m}))\cong\varprojlim R/\mathfrak{m}^{n}$.
    \begin{proof}
        Tenim que
        \begin{align*}
            &\Hom_{R}(E_{R}(R/\mathfrak{m}),E_{R}(R/\mathfrak{m}))\\
            \cong\,
            &\Hom_{R}(\varinjlim E_{R/\mathfrak{m}^{n}}(R/\mathfrak{m}),\varinjlim E_{R/\mathfrak{m}^{n}}(R/\mathfrak{m}))
            &\quad&\textrm{(Per \ref{lim-inj-envolvent-cos-residual})}\\
            \cong\,
            &\Hom_{R}(\varinjlim E_{R/\mathfrak{m}^{n}}(R/\mathfrak{m}), E_{R/\mathfrak{m}^{n}}(R/\mathfrak{m}))
            &\quad&\textrm{(Endomorfismes preserven la filtració)}\\
            \cong\,
            &\varprojlim\Hom_{R}(E_{R/\mathfrak{m}^{n}}(R/\mathfrak{m}),E_{R/\mathfrak{m}^{n}}(R/\mathfrak{m}))
            &\quad&\textrm{(Hom preserva els colímits)}\\
            \cong\,
            &\varprojlim R/\mathfrak{m}^{n}
            &\quad&\textrm{(Per \ref{artinian-local-isomorphism})}
        \end{align*}
        com volíem veure.
    \end{proof}
\end{teorema}
\subsection{$p$-grup de Prüfer}
Considerem $\mathbb{Z}[\frac{1}{p}]/\mathbb{Z}$. Es una comprovacó rutinària veure que $\mathbb{Z}[\frac{1}{p}]/\mathbb{Z}\cong\varinjlim\mathbb{Z}/(p^{n})$. Amb \ref{m-adic-isomorphism}, serà suficient veure que $\mathbb{Z}[\frac{1}{p}]/\mathbb{Z}$ és l'envolvent injectiu de $\mathbb{Z}/(p)$ sobre $\mathbb{Z}$.
\begin{definicio}
    Un $\mathbb{Z}$-mòdul $G$ és divisible si $\forall x\forall n((x\in G\land n\in\mathbb{N})\Rightarrow\exists y(y\in G\land ny=x))$.
\end{definicio}
\begin{proposicio}\label{divisible-implica-injectiu}
    Tot $\mathbb{Z}$-mòdul $J$ divisible és injectiu.
    \begin{proof}
        Siguin $A\subset B$ $\mathbb{Z}$-mòduls i $\varphi\in\Hom_{\mathbb{Z}}(A,J)$. Volem estendre $\varphi$ a un element de $\Hom_{\mathbb{Z}}(B,J)$. Sigui
        \begin{equation*}
            \mathcal{S}:=\{(A',\varphi')\in\Obj(\textrm{Mod}_{\mathbb{Z}})\times\Hom_{\mathbb{Z}}(A',J):A\subset A'\subset B\land\varphi'|_{A}=\varphi\}
        \end{equation*}
        conjunt parcialment ordenat per l'ordre $\leq$ definit per
        \begin{equation*}
            (A',\varphi')\leq(A'',\varphi''):\iff A'\subset A''\land\varphi''|_{A'}=\varphi'
        \end{equation*} $\mathcal{S}\neq\emptyset$, ja que $(A,\varphi)\in\mathcal{S}$. Considerem una cadena $\{(A_{i},\varphi_{i}):i\in\mathscr{I}\}$ de $\mathcal{S}$. Tenim que $(\bigcup_{i\in\mathscr{I}}A_{i},\varphi)\in\mathcal{S}$ és una cota superior de $\{(A_{i},\varphi_{i}):i\in\mathscr{I}\}$, on $\varphi\in\Hom_{\mathbb{Z}}(\bigcup_{i\in\mathscr{I}}A_{i},J)$ ve definida per $\varphi(x):=\varphi_{i}(x)$ si $x\in A_{i}$. Aleshores, pel lema de Zorn, $\mathcal{S}$ té un element maximal $(A',\varphi')\in\mathcal{S}$.\newline
        Volem veure que $A'=B$. Suposem que $A'\subsetneq B$. Sigui $x\in B-A'$. Suposem que $\forall n(n\in\mathbb{Z}\Rightarrow nx\notin A')$. Definim $\varphi''\in\Hom_{\mathbb{Z}}(A'+\mathbb{Z}x,J)$ per $\varphi''(a+nx):=\varphi(a)$. Tenim que $(A',\varphi')\leq(A'+\mathbb{Z}x,\varphi'')\in\mathcal{S}$, contradicció amb la maximalitat de $(A',\varphi')$. Suposem que $\exists n(n\in\mathbb{Z}\land nx\in A')$ A més, imposem que $n$ sigui mínima. Per divisibilitat de $J$, $\forall x(x\in A'\Rightarrow\exists y(y\in J\land ny=\varphi(nx)))$. Considerem $\varphi''\in\Hom_{\mathbb{Z}}(A'\oplus\mathbb{Z},J)$ definit per $\varphi''(a,m):=\varphi(a)+mny$. Considerem $\varphi_{0}\in\Hom_{\mathbb{Z}}(A'\oplus\mathbb{Z},B)$ definit per $\varphi_{0}(a,m):=a+mnx$. Si $(a,m)\in\ker{\varphi_{0}}$, $\varphi''(a,m)=\varphi(a)+mny=\varphi(a)+m\varphi(nx)=\varphi(a+mnx)=\varphi(0)=0$. Per tant, $\ker{\varphi_{0}}\subset\ker{\varphi''}$, d'on tenim la factorització
        \begin{equation*}
        \begin{tikzcd}
            A'\oplus\mathbb{Z}
            \arrow{rr}{\varphi''}
            \arrow[swap]{rd}{\varphi_{0}}
            &&J\\
            &A'+x(n)
            \arrow[dotted, swap]{ru}{\overline{\varphi}_{0}}
        \end{tikzcd}
        \end{equation*}
        $\overline{\varphi}_{0}\in\Hom_{\mathbb{Z}}(A'+x(n),J)$ definida per $\overline{\varphi}_{0}(a+mnx):=\varphi(a)+mnz$. Obtenim $(A',\varphi')\leq(A'+x(n),\overline{\varphi}_{0})\in\mathcal{S}$, contradicció amb la maximalitat de $(A',\varphi')$. Per tant, $A'=B$.
    \end{proof}
\end{proposicio}
El recíproc també és cert. És fàcil veure que $\mathbb{Z}[\frac{1}{p}]/\mathbb{Z}$ és $p$-divisible i, per tant, divisible. Com $\mathbb{Z}[\frac{1}{p}]/\mathbb{Z}$ és un $\mathbb{Z}$-mòdul, deduïm que $\mathbb{Z}[\frac{1}{p}]/\mathbb{Z}$ és injectiu per \ref{divisible-implica-injectiu}.\newline
A més, $\mathbb{Z}[\frac{1}{p}]/\mathbb{Z}$ és essencial sobre $\mathbb{Z}/(p)$ ($\cong\mathbb{Z}_{(p)}/(p)\mathbb{Z}_{(p)}$) ja que (recordem \ref{equivalent-ext-essencial})
\begin{equation*}
    p\Big(\sum_{j=0}^{i}a_{j}p^{-j}+\mathbb{Z}\Big)=\sum_{j=0}^{i-1}a_{j-1}p^{-j}+(p^{i})\in\mathbb{Z}/(p)
\end{equation*}
Per tant,
    \begin{align*}
        \Hom_{\mathbb{Z}}(\mathbb{Z}[\tfrac{1}{p}]/\mathbb{Z},\mathbb{Z}[\tfrac{1}{p}]/\mathbb{Z})
        &=\Hom_{\mathbb{Z}_{(p)}}(\mathbb{Z}[\tfrac{1}{p}]/\mathbb{Z},\mathbb{Z}[\tfrac{1}{p}]/\mathbb{Z})
        &\quad&\textrm{(Hom-sets coincideixen)}\\
        &\cong\varprojlim\mathbb{Z}_{(p)}/\big((p)\mathbb{Z}_{(p)}\big)^{n}
        &\quad&\textrm{($(\mathbb{Z}_{(p)},\mathbb{Z}_{(p)}/(p)\mathbb{Z}_{(p)})$ anell noetherià local i \ref{m-adic-isomorphism})}\\
        &\cong\varprojlim\mathbb{Z}/(p^{n})
        &\quad&\textrm{($\mathbb{Z}_{(p)}/\big((p)\mathbb{Z}_{(p)}\big)^{n}\cong\mathbb{Z}/(p^{n})$)}\\
        &\cong\mathbb{Z}_{p}
    \end{align*}
d'on resulta l'isomorfisme.