\documentclass[compress]{article}
\usepackage[utf8]{inputenc}
\usepackage{amsmath}
\usepackage{amssymb}
\usepackage{amsthm}
\usepackage{tikz-cd}
\usepackage{fancyhdr}
\usepackage[english]{babel}
\usepackage[document]{ragged2e}
\usepackage{appendix}

\usepackage[colorlinks = true,
linkcolor = blue,
urlcolor  = blue,
citecolor = blue,
anchorcolor = blue]{hyperref}
\usepackage[backend=biber,style=alphabetic,sorting=ynt]{biblatex}
\addbibresource{bibliografia.bib}

\usepackage{silence}
%\WarningFilter{latex}{Citation}
%\WarningFilter{latex}{Reference}
%\WarningFilter{latex}{There were undefined references}
\WarningFilter{newunicodechar}{Redefining Unicode character}
\WarningFilter{hyperref}{Token not allowed in a PDF string}
\overfullrule=5pt

\usepackage[margin=1in]{geometry}

\usepackage[T1]{fontenc}
\usepackage{newunicodechar}
\usepackage[mathscr]{euscript}
\usepackage{adjustbox}
\usepackage{mathrsfs}
\usepackage[scr=rsfs,cal=boondox]{mathalfa}

\DeclareMathOperator{\Har}{Har}
\DeclareMathOperator{\Hom}{Hom}
\DeclareMathOperator{\ord}{ord}
\DeclareMathOperator{\im}{im}
\DeclareMathOperator{\Obj}{Obj}

%% Definicions, proposicions, etc. %%
\newtheorem{teorema}{Teorema}[section]
\newtheorem{definicio}[teorema]{Definició}
\newtheorem{lema}[teorema]{Lema}
\newtheorem{proposicio}[teorema]{Proposició}
\newtheorem{corolari}[teorema]{Corol·lari}

%\usepackage{lineno}
%\linenumbers

\title{\large{\textsc{\textbf{Envolvent injectiu sobre cossos residuals}}}}
\author{\textsc{Jordi Cardiel}}
\date{}

\begin{document}
\maketitle
\begin{abstract}
    We study the notion of injective hull of a module, i.e. the minimal injective over the module. Then, we specialize to the context of local artinian and noetherian rings $R$ with residue field $\kappa=R/\mathfrak{m}$ and show that the ring of endomorphisms of the injective hull of $\kappa$ is isomorphic to the $\mathfrak{m}$-adic completition of $R$. As a consequence, we show that the ring of endomorphisms of the Prüfer $p$-group is isomorphic to the $p$-adic integers.
\end{abstract}
\tableofcontents
\text{ }\\
La motivació principal d'aquest treball és demostrar l'isomorfisme $\mathbb{Z}_{p}$ amb el conjunt d'endomorfismes del $p$-grup de Prüfer. El $p$-grup de Prüfer és defineix com la unió ascendent dels grups
\begin{equation*}
    \mathbb{Z}/(p)
    \subset\mathbb{Z}/(p^{2})
    \subset\mathbb{Z}/(p^{3})
    \subset\ldots
\end{equation*}
És a dir, el $p$-grup de Prüfer és $\varinjlim\mathbb{Z}/(p^{n})$. L'objectiu és desenvolupar unes eines (que són resultat de la teoria de Matlis per a injectius indescomponibles sobre un anell commutatiu noetherià) per tal de demostrar aquest isomorfisme en un context més general i veure com s'aplica al $p$-grup de Prüfer.
\section{Injectivitat}
En aquesta secció introduïm tota la maquinària d'àlgebra commutativa per arribar als resultats que desitgem.
\subsection{Mòduls injectius}
Recordem que $I$ $R$-mòdul és injectiu si per tot monomorfisme de $R$-mòduls $g:A\rightarrow B$ i per tot morfisme de $R$-mòduls $h:A\rightarrow I$ existeix un morfisme de $R$-mòduls $h':B\rightarrow I$ tal que el següent diagrama
\begin{center}
\begin{tikzcd}
    &&I
    &&\\
    \\
    0
    \arrow{rr}
    &&A
    \arrow[hook]{rr}{g}
    \arrow{uu}{h}
    &&B
    \arrow[swap,dotted]{uull}{h'}
\end{tikzcd}
\end{center}
és commutatiu.\newline
La injectivitat de $I$ es pot expressar com a una propietat del functor contravariant $\Hom_{R}(-,I):\textrm{Mod}_{R}\rightarrow\textrm{Mod}_{R}$, on $\textrm{Mod}_{R}$ denota la categoria de $R$-mòduls. En general, si
\begin{center}
\begin{tikzcd}
    0
    \arrow{r}
    &A
    \arrow{r}{g}
    &B
    \arrow{r}{f}
    &C
    \arrow{r}
    &0
\end{tikzcd}
\end{center}
és una successió exacta curta en $\textrm{Mod}_{R}$, aleshores, per tot $I$ $R$-mòdul
\begin{center}
\begin{tikzcd}
    0
    \arrow{rr}
    &&\Hom_{R}(C,I)
    \arrow{rr}{\Hom_{R}(f,I)}
    &&\Hom_{R}(B,I)
    \arrow{rr}{\Hom_{R}(g,I)}
    &&\Hom_{R}(A,I)
\end{tikzcd}
\end{center}
és exacte. La condició d'injectivitat ens diu que $\Hom_{R}(g,I)$ és exhaustiu, és a dir, que la successió curta
\begin{center}
\begin{tikzcd}
    0
    \arrow{rr}
    &&\Hom_{R}(C,I)
    \arrow{rr}{\Hom_{R}(f,I)}
    &&\Hom_{R}(B,I)
    \arrow{rr}{\Hom_{R}(g,I)}
    &&\Hom_{R}(A,I)
    \arrow{rr}
    &&0
\end{tikzcd}
\end{center}
és exacte. Deduïm que $I$ $R$-mòdul sigui injectiu és equivalent a que $\Hom_{R}(-,I)$ és exacte (és a dir, que la darrera successió curta és exacte).
\begin{lema}
    Sigui $R\rightarrow S$ un morfisme d'anells. Si $E$ és un $R$-mòdul injectiu, aleshores $\Hom_{R}(S,E)$ és un $S$-mòdul injectiu.
    \begin{proof}
        Sigui $S$-mòdul $M$ i, $M_{R}$, $M$ pensat com a $R$-mòdul. Tenim la correspondència
        \begin{center}
        \begin{tikzcd}
            \Hom_{R}(M_{R},E)
            \arrow[leftrightarrow]{rr}
            &&
            \Hom_{S}(M,\Hom_{R}(S,E)),
        \end{tikzcd}
        \end{center}
        donada per
        \begin{center}
        \begin{tikzcd}
            \alpha
            \arrow[mapsto]{rr}
            &&
            (m\mapsto(s\mapsto\alpha(sm)))
        \end{tikzcd}
        \end{center}
        amb inversa
        \begin{center}
        \begin{tikzcd}
            \beta
            \arrow[mapsto]{rr}
            &&
            (m\mapsto\beta(m)(1_{S}))
        \end{tikzcd}
        \end{center}
        Com $E$ és $R$-mòdul injectiu, $\Hom_{R}(-,E)$ és exacte. Per la correspondència,
        \begin{equation*}
            \Hom_{R}(-,E)=\Hom_{S}(-,\Hom_{R}(S,E)),
        \end{equation*}
        d'on deduïm que $\Hom_{S}(-,\Hom_{R}(S,E))$ és exacte i, per tant, $\Hom_{R}(S,E)$ és $S$-mòdul injectiu.
    \end{proof}
\end{lema}
\subsection{Extensió essencial}
Introduïm la següent definició que farem servir en aquesta secció.
\begin{definicio}
    Siguin $M\subset E$ $R$-mòduls. $M\subset E$ és una extensió essencial si tot $R$-submòdul no trivial d'$E$ interseca $M$ no trivialment.
\end{definicio}
\begin{lema}\label{equivalent-ext-essencial}
    Siguin $M\subset E$ $R$-mòduls. Són equivalents:
    \begin{enumerate}
        \item $M\subset E$ és una extensió essencial.
        \item $\forall x(x\in E-\{0\}\Rightarrow\exists r(r\in R\land rx\in M-\{0\}))$.
        \begin{proof}
            Suposem que $M\subset E$ és una extensió essencial. Sigui $x\in E-\{0\}$. Per essencialitat, $(x)\cap M\neq\{0\}$. Aleshores, $\exists r(r\in R\land rx\in M-\{0\})$.\newline
            Suposem que $\forall x(x\in E-\{0\}\Rightarrow\exists r(r\in R\land rx\in M-\{0\}))$. Sigui $E'\subset E$ $R$-submòdul no trivial d'$E$. Sigui $x\in E'-\{0\}$ (existeix per no trivialitat). Com $E'\subset E$, $x\in E-\{0\}$, d'on $\exists r(r\in R\land rx\in M-\{0\}))$ per hipòtesi. Com $rx\in E'\cap M$ i $rx\neq0$, deduïm que $E'\cap M\neq\{0\}$. Per tant, $E$ és una extensió essencial.
        \end{proof}
    \end{enumerate}
\end{lema}
Tenim el següent resultat sobre $R$-mòduls injectius i extensions essencials.
\begin{lema}\label{equivalent-injectiu}
    Sigui $I$ $R$-mòdul injectiu, $E\subset I$ $R$-submòdul. Són equivalents:
    \begin{enumerate}
        \item $E$ injectiu.
        \item Per tot $E\subset E'\subset I$ amb $E\subset E'$ extensió essencial, $E=E'$.
    \end{enumerate}
    \begin{proof}
        Suposem $E$ injectiu. Sigui $E'\subset I$ amb $E\subset E'$ extensió essencial. Per injectivitat d'$E$,
        \begin{center}
        \begin{tikzcd}
            &&E
            &&\\
            \\
            0
            \arrow{rr}
            &&E
            \arrow[hook]{rr}
            \arrow{uu}{id_{E}}
            &&E'
            \arrow[swap,dotted]{uull}{\exists\alpha}
        \end{tikzcd}
        \end{center}
        Suposem que $\ker{\alpha}\neq\{0\}$. Com $E\subset E'$ és una extensió essencial i $\ker{\alpha}$ $R$-submòdul d'$E'$ no trivial, $\ker{\alpha}\cap E\neq\{0\}$. Però, $\ker{\alpha}\cap E=\ker{id_{E}}$ i $\ker{id_{E}}=\{0\}$, contradicció. Per tant, $\ker{\alpha}=\{0\}$. Aleshores, $E'\cong E'/\ker{\alpha}\cong\im{\alpha}\subset E$, d'on deduïm que $E=E'$ per doble inclusió (ja que $E\subset E'$).\newline
        Suposem que per tot $E\subset E'\subset I$ amb $E\subset E'$ extensió essencial, $E=E'$. Siguin $M\subset N$ $R$-mòduls i $\varphi\in\Hom_{R}(M,E)$. Sigui
        \begin{equation*}
            \mathcal{S}:=\{(M',\varphi')\in\Obj(\textrm{Mod}_{R})\times\Hom_{R}(M',J):M\subset M'\subset N\land\varphi'|_{M}=\varphi\}
        \end{equation*}
        conjunt parcialment ordenat per l'ordre $\leq$ definit per
        \begin{equation*}
            (M',\varphi')\leq(M'',\varphi''):\iff M'\subset M''\land\varphi''|_{M'}=\varphi'
        \end{equation*}
        $\mathcal{S}\neq\emptyset$, ja que $(M,\varphi)\in\mathcal{S}$. Considerem una cadena $\{(M_{i},\varphi_{i}):i\in\mathscr{I}\}$ de $\mathcal{S}$. Tenim que $(\bigcup_{i\in\mathscr{I}}M_{i},\varphi)\in\mathcal{S}$ és una cota superior de $\{(M_{i},\varphi_{i}):i\in\mathscr{I}\}$, on $\varphi\in\Hom_{R}(\bigcup_{i\in\mathscr{I}}A_{i},J)$ ve definida per $\varphi(x):=\varphi_{i}(x)$ si $x\in M_{i}$. Aleshores, pel lema de Zorn, $\mathcal{S}$ té un element maximal $(M',\varphi')\in\mathcal{S}$.\newline
        Sigui $\iota:E\hookrightarrow I$ la inclusió. Per injectivitat d'$I$,
        \begin{center}
        \begin{tikzcd}
            &&I
            &&\\
            \\
            0
            \arrow{rr}
            &&M'
            \arrow[hook]{rr}
            \arrow{uu}{\iota\circ\varphi'}
            &&N
            \arrow[swap,dotted]{uull}{\exists\psi}
        \end{tikzcd}
        \end{center}
        Suposem que $\psi(N)\not\subset E$. Tenim $E\subsetneq E+\psi(N)\subset I$, d'on $E\subset E+\psi(N)$ no és essencial. Aleshores, existeix $K\subset E+\psi(N)$ no trivial tal que $K\cap E=\{0\}$. Com $M'\subset\psi^{-1}(E)\subsetneq\psi^{-1}(E+K)$,
        \begin{equation*}
            \pi\circ\psi|_{\psi^{-1}(E+K)}:\psi^{-1}(E+K)\rightarrow E+K\twoheadrightarrow E
        \end{equation*}
        és tal que $(\pi\circ\psi|_{\psi^{-1}(E+K)})|_{M'}=\varphi'$ i $\psi^{-1}(E+K)\subset N$. Aleshores, $(M',\varphi')\leq(\psi^{-1}(E+K),\pi\circ\psi|_{\psi^{-1}(E+K)})\in\mathcal{S}$, contradicció amb la maximalitat de $(M',\varphi')$.\newline
        Per tant, $\psi(N)\subset E$. Tenim que $\psi:N\rightarrow\psi(N)\hookrightarrow E$ i $\psi|_{M}=(\iota\circ\varphi')|_{M}=\varphi'|_{M}=\varphi$, d'on deduïm que $E$ és injectiu ($\psi\in\Hom_{R}(N,E)$ estén $\varphi\in\Hom_{R}(M,E)$).
    \end{proof}
\end{lema}
Recordem que un $R$-mòdul $M$ és noetherià si i només si tot $R$-submòdul de $M$ és finitament generat. Denotem per $(0:_{R}x)$ l'aniquil·lador de $x\in R$; similarment, $(0:_{R}I)$ és l'aniqul·lador de $I$, $I\subset R$ ideal.
\begin{lema}\label{lema-tecnic-noetheria-injectiu}
    Sigui $R$ anell noetherià, $I$ $R$-mòdul injectiu.
    \begin{enumerate}
        \item Sigui $f\in R$. Aleshores, $\bigcup_{n>0}(0:_{I}f^{n})$ $R$-submòdul injectiu de $I$.
        \begin{proof}
            Sigui $E'\subset I$ amb $\bigcup_{n>0}(0:_{I}f^{n})\subset E'$ extensió essencial. Suposem que $\bigcup_{n>0}(0:_{I}f^{n})\newline\subsetneq E'$ i volem arribar a contradicció. Aleshores, $\exists x(x\in E'-\bigcup_{n>0}(0:_{I}f^{n}))$. Considerem l'ideal $\bigcup_{n>0}(0:_{R}f^{n}x)\subset R$. Com $R$ és noetherià,
            \begin{equation*}
                \exists g_{1}\ldots\exists g_{t}\Big(g_{1},\ldots,g_{t}\in R\land\bigcup_{n>0}(0:_{R}f^{n}x)=(g_{1},\ldots,g_{t})\Big)
            \end{equation*}
            Com $g_{1},\ldots,g_{t}\in\bigcup_{n>0}(0:_{R}f^{n}x)$, $\exists n_{1}\ldots\exists n_{t}(n_{1},\ldots,n_{t}>0\land\forall i(g_{i}f^{n_{i}}x=0))$. Definim
            \begin{equation*}
                x':=f^{\max\{n_{i}\}}x\in E'-\bigcup_{n>0}(0:_{I}f^{n})
            \end{equation*}
            Sigui $r\in(g_{1},\ldots,g_{t})$. Aleshores, $\exists a_{1}\ldots\exists a_{t}(a_{1},\ldots,a_{t}\in R\land r=\sum_{i=1}^{t}a_{i}g_{i})$. Per tant,
            \begin{align*}
                rx'
                &=\sum_{i=1}^{t}a_{i}(g_{i}f^{\max\{n_{i}\}}x)
                &\quad&\textrm{(Per definició de $x'$)}\\
                &=\sum_{i=1}^{t}a_{i}0
                \,(=0)
                &\quad&\textrm{($\forall i(g_{i}f^{n_{i}}x=0)$)}
            \end{align*}
            Per tant, $r\in(0:_{R}x')$ i $(g_{1},\ldots,g_{t})\subset(0:_{R}x')$. Com
            \begin{align*}
                (0:_{R}x')
                &=(0:_{R}f^{\max\{n_{i}\}}x)
                &\quad&\textrm{(Per definició de $x'$)}\\
                &\subset\bigcup_{n>0}(0:_{R}f^{n}x)
                &\quad&\textrm{($\max\{n_{i}\}>0$)}
            \end{align*}
            deduïm per doble inclusió que
            \begin{equation*}
                (0:_{R}x')
                =\bigcup_{n>0}(0:_{R}f^{n}x)
            \end{equation*}
            Sigui $r\in(g_{1},\ldots,g_{t})$. Aleshores, $\exists a_{1}\ldots\exists a_{t}(a_{1},\ldots,a_{t}\in R\land r=\sum_{i=1}^{t}a_{i}g_{i})$, d'on
            \begin{equation*}
                r(f^{n}x')=\sum_{i=1}^{t}a_{i}g_{i}f^{n}f^{\max\{n_{i}\}}x=0
                \implies
                r\in\bigcap_{n>0}(0:_{R}f^{n}x')\subset\bigcup_{n>0}(0:_{R}f^{n}x')
            \end{equation*}
            Per tant, $(g_{1},\ldots,g_{t})\subset\bigcup_{n>0}(0:_{R}f^{n}x')$. Com
            \begin{align*}
                \bigcup_{n>0}(0:_{R}f^{n}x')
                &=\bigcup_{n>0}(0:_{R}f^{n+\max\{n_{i}\}}x)\\
                &\subset\bigcup_{n>0}(0:_{R}f^{n}x)
            \end{align*}
            deduïm que
            \begin{equation*}
                \bigcup_{n>0}(0:_{R}f^{n}x)
                =\bigcup_{n>0}(0:_{R}f^{n}x')
            \end{equation*}
            Per transitivitat de $=$,
            \begin{equation*}
                (0:_{R}x')
                =\bigcup_{n>0}(0:_{R}f^{n}x')
            \end{equation*}
            Sigui $y\in(x')\cap\bigcup_{n>0}(0:_{I}f^{n})$. Aleshores, $\exists n(n>0\land yf^{n}=0)$ i $\exists r'(r'\in R\land r'x'=y)$, d'on
            \begin{align*}
                r'f^{n}x'
                &=yf^{n}
                &\quad&\textrm{($r'x'=y$)}\\
                &=0
                &\quad&\textrm{($yf^{n}=0$)}
            \end{align*}
            Aleshores,
            \begin{align*}
                r'\in\bigcup_{n>0}(0:_{R}f^{n}x')
                &\implies
                r'\in(0:_{R}x')
                &\quad&\textrm{\Big($(0:_{R}x')
                =\bigcup_{n>0}(0:_{R}f^{n}x')$\Big)}\\
                &\implies
                y=r'x'=0
                &\quad&\textrm{($r'x'=y$)}
            \end{align*}
            Per tant, $(x')\cap\bigcup_{n>0}(0:_{I}f^{n})=\{0\}$, d'on deduïm que $\bigcup_{n>0}(0:_{I}f^{n})\subset E'$ no és una extensió essencial, contradicció. Per tant, $\bigcup_{n>0}(0:_{I}f^{n})=E'$ i, per \ref{equivalent-injectiu}, $\bigcup_{n>0}(0:_{I}f^{n})$ és $R$-submòdul injectiu de $I$.
        \end{proof}
        \item Sigui $J\subset R$ ideal. Aleshores, $\bigcup_{n>0}(0:_{I}J^{n})$ $R$-submòdul injectiu de $I$.
        \begin{proof}
            Com $R$ és noetherià, $\exists f_{1}\ldots\exists f_{t}(f_{1},\ldots f_{t}\in R\land J=(f_{1},\ldots,f_{t}))$. Aleshores, com
            \begin{equation*}
                \bigcup_{n>0}(0:_{I}J^{n})
                =\bigcup_{n>0}(0:_{\bigcup_{n>0}(0:_{\cdots_{(\bigcup_{n>0}(0:_{(\bigcup_{n>0}(0:_{I}f_{1}^{n})}f_{2}^{n}))}\cdots}f_{t-1}^{n})}f_{t}^{n})
            \end{equation*}
            ens reduïm al cas anterior i procedim per inducció.
        \end{proof}
    \end{enumerate}
\end{lema}
\subsection{Envolvent injectiu}
Introduïm la noció d'envolvent injectiu, que utilitzarem en la següent secció. La idea és barrejar les dos nocions anteriors.
\begin{definicio}
    Siguin $M\subset I$ $R$-mòduls. Direm que $I$ és l'envolvent injectiu de $M$ si $I$ és injectiu i $M\subset I$ és una extensió essencial.
\end{definicio}
Enunciem el següent resultat sense demostració, el qual dona altres caracteritzacions i la unicitat llevat isomorfisme de l'envolvent injectiu.
\begin{proposicio}
    Sigui $M\subset I$ $R$-mòduls. Són equivalents:
    \begin{enumerate}
        \item $M\subset I$ és l'extensió essencial maximal.
        \item $I$ és injectiu i $M\subset I$ és una extensió essencial.
        \item $I$ és injectiu minimal sobre $M$.
    \end{enumerate}
    A més, tot $R$-mòdul té un envolvent injectiu i donats $I,I'$ envolvents injectius de $M$, existeix un isomorfisme $g:I\rightarrow I'$ tal que $g|_{M}=id_{M}$.
    \begin{proof}
        Veure \cite{Lam1999}, 3.29., 3.30. i 3.32.
    \end{proof}
\end{proposicio}
Escriurem per $E_{R}(M)$ l'envolvent injectiu del $R$-mòdul $M$. Necessitarem el següent resultat per la següent resultat per la següent secció.
\begin{lema}\label{env-inj-aniquillador}
    Sigui $R$ anell, $\mathfrak{a}$ ideal de $R$ i $M$ $R$-mòdul tal que $\mathfrak{a}M=\{0\}$. Aleshores, $E_{R/\mathfrak{a}}(M)\cong(0:_{E}\mathfrak{a})$.
    \begin{proof}
        Tenim que $\mathfrak{a}M,\mathfrak{a}(0:_{E}\mathfrak{a})=\{0\}$. Aleshores, considerem $M$ i $(0:_{E_{R}(M)}\mathfrak{a})$ com $R/\mathfrak{a}$-mòduls. Com $M\subset E_{R}(M)$ i $\mathfrak{a}M=\{0\}$, deduïm que $M\subset(0:_{E_{R}(M)}\mathfrak{a})$. Com tot $R/\mathfrak{a}$-submodul de $(0:_{E_{R}(M)}\mathfrak{a})$ és $R$-submòdul de $E_{R}(M)$, necessàriament $M\subset(0:_{E_{R}(M)}\mathfrak{a})$ és una extensió essencial.\newline
        Veiem que podem completar el diagrama de $R/\mathfrak{a}$-mòduls
        \begin{center}
        \begin{tikzcd}
            &&(0:_{E_{R}(M)}\mathfrak{a})
            &&\\
            \\
            0
            \arrow{rr}
            &&A
            \arrow[hook]{rr}{g}
            \arrow{uu}{h}
            &&B
            %\arrow[swap,dotted]{uull}
        \end{tikzcd}
        \end{center}
        En efecte, si els pensem com $R$-mòduls, per injectivitat de $E_{R}(M)$ tenim que
        \begin{center}
        \begin{tikzcd}
            &&E_{R}(M)
            &&\\\\
            &&(0:_{E_{R}(M)}\mathfrak{a})
            \arrow[hook]{uu}{\iota}
            &&\\\\
            0
            \arrow{rr}
            &&A
            \arrow[hook]{rr}{g}
            \arrow{uu}{h}
            &&B
            \arrow[swap,dotted]{uuuull}{\exists(\iota\circ h)'}
        \end{tikzcd}
        \end{center}
        commuta. De la commutativitat es comprova que $\im{(\iota\circ h')}\subset(0:_{E_{R}(M)}\mathfrak{a})$, d'on deduïm que $(0:_{E_{R}(M)}\mathfrak{a})$ és injectiu.
    \end{proof}
\end{lema}

\section{Envolvent injectiu sobre cossos residuals}
Recordem que un anell $R$ és local si només té un ideal maximal $\mathfrak{m}$. En aquest cas, diem que $R/\mathfrak{m}$ és el cos residual de $R$.
\subsection{Resultats sobre anells artinians locals}
Sigui $M$ $R$-mòdul ($R$ no necessàriament local). Definim la longitud de $M$ com
\begin{equation*}
    \ell_{R}(M):=\sup\{n\in\mathbb{N}:\exists(0\subsetneq M_{1}\subsetneq\ldots\subsetneq M_{n}=M)\}
\end{equation*}
Es pot comprovar que $\ell_{R}$ és una funció additiva, és a dir, donada una successió exacta curta a $\textrm{Mod}_{R}$
\begin{center}
    \begin{tikzcd}
    0
    \arrow{r}
    &A
    \arrow{r}
    &B
    \arrow{r}
    &C
    \arrow{r}
    &0
\end{tikzcd}
\end{center}
tenim que $\ell_{R}(B)=\ell_{R}(A)+\ell_{R}(C)$.\newline
Considerarem en aquesta secció anells artinians locals. D'aquestes condicions, es dedueix que l'anell artinià local és noetherià i té $R$ té longitud finita com a $R$-mòdul.
\begin{lema}\label{artinian-local-lenght}
    Sigui $(R,\mathfrak{m})$ anell artinià local, $M$ $R$-mòdul finitament generat. Aleshores,
    \begin{equation*}
        \ell_{R}(M)=\ell_{R}(\Hom_{R}(M,E_{R}(R/\mathfrak{m})))
    \end{equation*}
    \begin{proof}
        Tenim que
        \begin{align*}
            \Hom_{R}(R/\mathfrak{m},E_{R}(R/\mathfrak{m}))
            &\cong E_{R/\mathfrak{m}}(R/\mathfrak{m})
            &\quad&\textrm{(Isomorfisme via $\varphi\mapsto\varphi(1_{R}+\mathfrak{m})$)}\\
            &\cong R/\mathfrak{m}
            &\quad&\textrm{(Tot mòdul sobre un cos és injectiu i $R/\mathfrak{m}$ és minimal)}
        \end{align*}
        Aleshores, 
        \begin{equation*}
            \ell_{R}(R/\mathfrak{m})=\ell_{R}(\Hom_{R}(M,E_{R}(R/\mathfrak{m})))
        \end{equation*}
        Procedim per inducció en $\ell_{R}(M)$, $M$ $R$-mòdul finitament generat. Tenim que existeix $\pi:M\twoheadrightarrow R/\mathfrak{m}$ morfisme de $R$-mòduls exhaustiu. Considerem la successió exacta curta
        \begin{center}
        \begin{tikzcd}
            0
            \arrow{r}
            &\ker{\pi}
            \arrow{r}{\iota}
            &M
            \arrow{r}{\pi}
            &R/\mathfrak{m}
            \arrow{r}
            &0
        \end{tikzcd}
        \end{center}
        Com $E_{R}(R/\mathfrak{m})$ és injectiu, obtenim la successió exacta curta
        \begin{center}
        \begin{tikzcd}
            0
            \arrow{r}
            &\Hom_{R}(R/\mathfrak{m},E_{R}(R/\mathfrak{m}))
            \arrow{r}{\pi_{*}}
            &\Hom_{R}(M,E_{R}(R/\mathfrak{m}))
            \arrow{r}{\iota_{*}}
            &\Hom_{R}(\ker{\pi},E_{R}(R/\mathfrak{m}))
            \arrow{r}
            &0
        \end{tikzcd}
        \end{center}
        De les darreres successions exactes curtes, deduïm que
        \begin{align*}
            &\ell_{R}(\Hom_{R}(M,E_{R}(R/\mathfrak{m})))\\
            =\,
            &\ell_{R}(\Hom_{R}(R/\mathfrak{m},E_{R}(R/\mathfrak{m})))
            +\Hom_{R}(\ker{\pi},E_{R}(R/\mathfrak{m}))
            &\quad&\textrm{(Per additivitat de $\ell_{R}$)}\\
            =\,
            &\ell_{R}(R/\mathfrak{m})
            +\ell_{R}(\ker{\pi})
            &\quad&\textrm{(Hipòtesi d'inducció)}\\
            =\,
            &\ell_{R}(M)
            &\quad&\textrm{(Per additivitat de $\ell_{R}$)}
        \end{align*}
        com volíem veure.
    \end{proof}
\end{lema}
\begin{corolari}\label{artinian-local-isomorphism}
    Sigui $(R,\mathfrak{m})$ anell artinià local, $M$ $R$-mòdul finitament generat. Aleshores,
    \begin{equation*}
        M\cong\Hom_{R}(\Hom_{R}(M,E_{R}(R/\mathfrak{m})),E_{R}(R/\mathfrak{m}))
    \end{equation*}
    En particular, $R\cong\Hom_{R}(E_{R}(R/\mathfrak{m}),E_{R}(R/\mathfrak{m}))$.
    \begin{proof}
        Per tot $m\in M-\{0\}$, existeix un morfisme de $R$-mòduls no trivial $h_{m}:(m)\rightarrow R/\mathfrak{m}$. Com $E_{R}(R/\mathfrak{m})$ injectiu i $R/\mathfrak{m}\subset E_{R}(R/\mathfrak{m})$,
        \begin{center}
        \begin{tikzcd}
            &&E_{R}(R/\mathfrak{m})
            &&\\\\
            &&R/\mathfrak{m}
            \arrow[hook]{uu}{\iota'}
            &&\\\\
            0
            \arrow{rr}
            &&(m)
            \arrow[hook]{rr}{\iota}
            \arrow{uu}{h_{m}}
            &&M
            \arrow[swap,dotted]{uuuull}{\exists(\iota'\circ h_{m})'}
        \end{tikzcd}
        \end{center}
        Per tant, el morfisme $M\rightarrow\Hom_{R}(\Hom_{R}(M,E_{R}(R/\mathfrak{m})),E_{R}(R/\mathfrak{m}))$ definit via $m\mapsto((\iota'\circ h_{m})'\mapsto(\iota'\circ h_{m})'(m))$ esta ben definit i és injectiu (ja que per construcció té nucli trivial). A més, com
        \begin{align*}
            \ell_{R}(\Hom_{R}(\Hom_{R}(M,E_{R}(R/\mathfrak{m})),E_{R}(R/\mathfrak{m})))
            &=\ell_{R}(\Hom_{R}(M,E_{R}(R/\mathfrak{m})))
            &\quad&\textrm{(Per \ref{artinian-local-lenght})}\\
            &=\ell_{R}(M)
            &\quad&\textrm{(Per \ref{artinian-local-lenght})}
        \end{align*}
        deduïm que és isomorfisme. El darrer isomorfisme resulta de
        \begin{align*}
            R
            &\cong\Hom_{R}(\Hom_{R}(R,E_{R}(R/\mathfrak{m})),E_{R}(R/\mathfrak{m}))\\
            &\cong
            \Hom_{R}(E_{R}(R/\mathfrak{m}),E_{R}(R/\mathfrak{m}))
            &\quad&\textrm{($E_{R}(R/\mathfrak{m})\cong\Hom_{R}(R,E_{R}(R/\mathfrak{m}))$)}
        \end{align*}
        on el darrer isomorfisme és general per tot $R$-mòdul.
    \end{proof}
\end{corolari}
\subsection{Resultats sobre anells noetherians locals}
Volem aplicar \ref{env-inj-aniquillador} al ideal $\mathfrak{m}^{n}$.
\begin{proposicio}
    Sigui $(R,\mathfrak{m})$ un anell noetherià local. Aleshores,
    \begin{enumerate}
        \item $E_{R/\mathfrak{m}^{n}}(R/\mathfrak{m})\cong(0:_{E_{R}(R/\mathfrak{m})}\mathfrak{m}^{n})$.
        \begin{proof}
            $\mathfrak{m}^{n}$ és un ideal de $R$ i $\mathfrak{m}^{n}(R/\mathfrak{m})=\{0\}$. Per \ref{env-inj-aniquillador}, $E_{R/\mathfrak{m}^{n}}(R/\mathfrak{m})\cong(0:_{E_{R}(R/\mathfrak{m})}\mathfrak{m}^{n})$.
        \end{proof}
        \item $E_{R}(R/\mathfrak{m})=\bigcup_{n>0}(0:_{E_{R}(R/\mathfrak{m})}\mathfrak{m}^{n})$.
        \begin{proof}
        Com $R$ és noetherià i $E_{R}(R/\mathfrak{m})$ és un $R$-mòdul injectiu, $\bigcup_{n>0}(0:_{E_{R}(R/\mathfrak{m})}\mathfrak{m}^{n})$ és un $R$-submòdul injectiu d'$E_{R}(R/\mathfrak{m})$ per \ref{lema-tecnic-noetheria-injectiu}. Fixem-nos que
        \begin{equation*}
            R/\mathfrak{m}\subset(0:_{E_{R}(R/\mathfrak{m})}\mathfrak{m}^{n})\subset E_{R}(R/\mathfrak{m})
        \end{equation*}
        Com $E_{R}(R/\mathfrak{m})$ és l'envolvent injectiu de $R/\mathfrak{m}$, $E_{R}(R/\mathfrak{m})$ és l'injectiu més petit que conté $R/\mathfrak{m}$, d'on resulta $E_{R}(R/\mathfrak{m})\subset\bigcup_{n>0}(0:_{E_{R}(R/\mathfrak{m})}\mathfrak{m}^{n})$ i, per tant, $E_{R}(R/\mathfrak{m})=\bigcup_{n>0}(0:_{E_{R}(R/\mathfrak{m})}\mathfrak{m}^{n})$.
    \end{proof}
    \end{enumerate}
\end{proposicio}
\begin{corolari}\label{lim-inj-envolvent-cos-residual}
    Sigui $(R,\mathfrak{m})$ anell noetherià local. Aleshores, $\varinjlim E_{R/\mathfrak{m}^{n}}(R/\mathfrak{m})\cong E_{R}(R/\mathfrak{m})$.
\end{corolari}
En particular, tenim la filtració $\{E_{R/\mathfrak{m}^{n}}(R/\mathfrak{m}):n\in\mathbb{N}\}$. Els endomorfismes preserven la filtració, ja que
\begin{equation*}
    \im\Big(\varphi:(0:_{E_{R}(R/\mathfrak{m})}\mathfrak{m}^{n})\rightarrow\bigcup_{n>0}(0:_{E_{R}(R/\mathfrak{m})}\mathfrak{m}^{n})\Big)
    \subset(0:_{E_{R}(R/\mathfrak{m})}\mathfrak{m}^{n})
\end{equation*}
En efecte, de forma heurística, tenim que $f((0:_{E_{R}(R/\mathfrak{m})}\mathfrak{m}^{n}))\mathfrak{m}^{n}=f((0:_{E_{R}(R/\mathfrak{m})}\mathfrak{m}^{n})\mathfrak{m}^{n})=f(\{0\})=\{0\}$, d'on es dedueix la inclusió. Aquesta preservació simplifica la demostració del resultat següent.
\begin{teorema}\label{m-adic-isomorphism}
    Sigui $(R,\mathfrak{m})$ anell noetherià local. Aleshores, $\Hom_{R}(E_{R}(R/\mathfrak{m}),E_{R}(R/\mathfrak{m}))\cong\varprojlim R/\mathfrak{m}^{n}$.
    \begin{proof}
        Tenim que
        \begin{align*}
            &\Hom_{R}(E_{R}(R/\mathfrak{m}),E_{R}(R/\mathfrak{m}))\\
            \cong\,
            &\Hom_{R}(\varinjlim E_{R/\mathfrak{m}^{n}}(R/\mathfrak{m}),\varinjlim E_{R/\mathfrak{m}^{n}}(R/\mathfrak{m}))
            &\quad&\textrm{(Per \ref{lim-inj-envolvent-cos-residual})}\\
            \cong\,
            &\Hom_{R}(\varinjlim E_{R/\mathfrak{m}^{n}}(R/\mathfrak{m}), E_{R/\mathfrak{m}^{n}}(R/\mathfrak{m}))
            &\quad&\textrm{(Endomorfismes preserven la filtració)}\\
            \cong\,
            &\varprojlim\Hom_{R}(E_{R/\mathfrak{m}^{n}}(R/\mathfrak{m}),E_{R/\mathfrak{m}^{n}}(R/\mathfrak{m}))
            &\quad&\textrm{(Hom preserva els colímits)}\\
            \cong\,
            &\varprojlim R/\mathfrak{m}^{n}
            &\quad&\textrm{(Per \ref{artinian-local-isomorphism})}
        \end{align*}
        com volíem veure.
    \end{proof}
\end{teorema}
\subsection{$p$-grup de Prüfer}
Considerem $\mathbb{Z}[\frac{1}{p}]/\mathbb{Z}$. Es una comprovacó rutinària veure que $\mathbb{Z}[\frac{1}{p}]/\mathbb{Z}\cong\varinjlim\mathbb{Z}/(p^{n})$. Amb \ref{m-adic-isomorphism}, serà suficient veure que $\mathbb{Z}[\frac{1}{p}]/\mathbb{Z}$ és l'envolvent injectiu de $\mathbb{Z}/(p)$ sobre $\mathbb{Z}$.
\begin{definicio}
    Un $\mathbb{Z}$-mòdul $G$ és divisible si $\forall x\forall n((x\in G\land n\in\mathbb{N})\Rightarrow\exists y(y\in G\land ny=x))$.
\end{definicio}
\begin{proposicio}\label{divisible-implica-injectiu}
    Tot $\mathbb{Z}$-mòdul $J$ divisible és injectiu.
    \begin{proof}
        Siguin $A\subset B$ $\mathbb{Z}$-mòduls i $\varphi\in\Hom_{\mathbb{Z}}(A,J)$. Volem estendre $\varphi$ a un element de $\Hom_{\mathbb{Z}}(B,J)$. Sigui
        \begin{equation*}
            \mathcal{S}:=\{(A',\varphi')\in\Obj(\textrm{Mod}_{\mathbb{Z}})\times\Hom_{\mathbb{Z}}(A',J):A\subset A'\subset B\land\varphi'|_{A}=\varphi\}
        \end{equation*}
        conjunt parcialment ordenat per l'ordre $\leq$ definit per
        \begin{equation*}
            (A',\varphi')\leq(A'',\varphi''):\iff A'\subset A''\land\varphi''|_{A'}=\varphi'
        \end{equation*} $\mathcal{S}\neq\emptyset$, ja que $(A,\varphi)\in\mathcal{S}$. Considerem una cadena $\{(A_{i},\varphi_{i}):i\in\mathscr{I}\}$ de $\mathcal{S}$. Tenim que $(\bigcup_{i\in\mathscr{I}}A_{i},\varphi)\in\mathcal{S}$ és una cota superior de $\{(A_{i},\varphi_{i}):i\in\mathscr{I}\}$, on $\varphi\in\Hom_{\mathbb{Z}}(\bigcup_{i\in\mathscr{I}}A_{i},J)$ ve definida per $\varphi(x):=\varphi_{i}(x)$ si $x\in A_{i}$. Aleshores, pel lema de Zorn, $\mathcal{S}$ té un element maximal $(A',\varphi')\in\mathcal{S}$.\newline
        Volem veure que $A'=B$. Suposem que $A'\subsetneq B$. Sigui $x\in B-A'$. Suposem que $\forall n(n\in\mathbb{Z}\Rightarrow nx\notin A')$. Definim $\varphi''\in\Hom_{\mathbb{Z}}(A'+\mathbb{Z}x,J)$ per $\varphi''(a+nx):=\varphi(a)$. Tenim que $(A',\varphi')\leq(A'+\mathbb{Z}x,\varphi'')\in\mathcal{S}$, contradicció amb la maximalitat de $(A',\varphi')$. Suposem que $\exists n(n\in\mathbb{Z}\land nx\in A')$ A més, imposem que $n$ sigui mínima. Per divisibilitat de $J$, $\forall x(x\in A'\Rightarrow\exists y(y\in J\land ny=\varphi(nx)))$. Considerem $\varphi''\in\Hom_{\mathbb{Z}}(A'\oplus\mathbb{Z},J)$ definit per $\varphi''(a,m):=\varphi(a)+mny$. Considerem $\varphi_{0}\in\Hom_{\mathbb{Z}}(A'\oplus\mathbb{Z},B)$ definit per $\varphi_{0}(a,m):=a+mnx$. Si $(a,m)\in\ker{\varphi_{0}}$, $\varphi''(a,m)=\varphi(a)+mny=\varphi(a)+m\varphi(nx)=\varphi(a+mnx)=\varphi(0)=0$. Per tant, $\ker{\varphi_{0}}\subset\ker{\varphi''}$, d'on tenim la factorització
        \begin{equation*}
        \begin{tikzcd}
            A'\oplus\mathbb{Z}
            \arrow{rr}{\varphi''}
            \arrow[swap]{rd}{\varphi_{0}}
            &&J\\
            &A'+x(n)
            \arrow[dotted, swap]{ru}{\overline{\varphi}_{0}}
        \end{tikzcd}
        \end{equation*}
        $\overline{\varphi}_{0}\in\Hom_{\mathbb{Z}}(A'+x(n),J)$ definida per $\overline{\varphi}_{0}(a+mnx):=\varphi(a)+mnz$. Obtenim $(A',\varphi')\leq(A'+x(n),\overline{\varphi}_{0})\in\mathcal{S}$, contradicció amb la maximalitat de $(A',\varphi')$. Per tant, $A'=B$.
    \end{proof}
\end{proposicio}
El recíproc també és cert. És fàcil veure que $\mathbb{Z}[\frac{1}{p}]/\mathbb{Z}$ és $p$-divisible i, per tant, divisible. Com $\mathbb{Z}[\frac{1}{p}]/\mathbb{Z}$ és un $\mathbb{Z}$-mòdul, deduïm que $\mathbb{Z}[\frac{1}{p}]/\mathbb{Z}$ és injectiu per \ref{divisible-implica-injectiu}.\newline
A més, $\mathbb{Z}[\frac{1}{p}]/\mathbb{Z}$ és essencial sobre $\mathbb{Z}/(p)$ ($\cong\mathbb{Z}_{(p)}/(p)\mathbb{Z}_{(p)}$) ja que (recordem \ref{equivalent-ext-essencial})
\begin{equation*}
    p\Big(\sum_{j=0}^{i}a_{j}p^{-j}+\mathbb{Z}\Big)=\sum_{j=0}^{i-1}a_{j-1}p^{-j}+(p^{i})\in\mathbb{Z}/(p)
\end{equation*}
Per tant,
    \begin{align*}
        \Hom_{\mathbb{Z}}(\mathbb{Z}[\tfrac{1}{p}]/\mathbb{Z},\mathbb{Z}[\tfrac{1}{p}]/\mathbb{Z})
        &=\Hom_{\mathbb{Z}_{(p)}}(\mathbb{Z}[\tfrac{1}{p}]/\mathbb{Z},\mathbb{Z}[\tfrac{1}{p}]/\mathbb{Z})
        &\quad&\textrm{(Hom-sets coincideixen)}\\
        &\cong\varprojlim\mathbb{Z}_{(p)}/\big((p)\mathbb{Z}_{(p)}\big)^{n}
        &\quad&\textrm{($(\mathbb{Z}_{(p)},\mathbb{Z}_{(p)}/(p)\mathbb{Z}_{(p)})$ anell noetherià local i \ref{m-adic-isomorphism})}\\
        &\cong\varprojlim\mathbb{Z}/(p^{n})
        &\quad&\textrm{($\mathbb{Z}_{(p)}/\big((p)\mathbb{Z}_{(p)}\big)^{n}\cong\mathbb{Z}/(p^{n})$)}\\
        &\cong\mathbb{Z}_{p}
    \end{align*}
d'on resulta l'isomorfisme.
\printbibliography
\end{document}
