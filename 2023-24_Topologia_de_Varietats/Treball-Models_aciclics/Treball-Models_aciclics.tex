\documentclass[compress]{article}
\usepackage[utf8]{inputenc}
\usepackage{amsmath}
\usepackage{amssymb}
\usepackage{amsthm}
\usepackage{tikz-cd}
\usepackage{fancyhdr}
\usepackage[catalan]{babel}
\usepackage[document]{ragged2e}
\usepackage{appendix}

\usepackage{silence}
%\WarningFilter{latex}{Citation}
%\WarningFilter{latex}{Reference}
%\WarningFilter{latex}{There were undefined references}
\WarningFilter{newunicodechar}{Redefining Unicode character}
\WarningFilter{hyperref}{Token not allowed in a PDF string}
\overfullrule=5pt

\usepackage[margin=1in]{geometry}

\usepackage[T1]{fontenc}
\usepackage{newunicodechar}
\usepackage[mathscr]{euscript}
\usepackage{adjustbox}

%% Definicions, proposicions, etc. %%
\newtheorem{definicio}{Definició}
\newtheorem{lema}{Lema}[section]
\newtheorem{proposicio}{Proposició}[section]
\newtheorem{corolari}{Corol·lari}
\newtheorem{teorema}{Teorema}
\newtheorem{problema}{\color{blue}Problema}[section]
\newtheorem*{claim}{Claim}
\newtheorem{enunciat}{}
\theoremstyle{definition}
\newtheorem{idea}{\color{gray}Idea}
\newtheorem{exemple}{Exemple}

\fancyhead[L]{\textbf{100114. Topologia de varietats.}\\
\textbf{Treball Final de Curs. Models acíclics.}}

\fancyhead[R]{\textit{Jordi Cardiel}}

% Letter fonts
\usepackage{mathrsfs}
\usepackage[scr=rsfs,cal=boondox]{mathalfa}

%% Operadors pel math mode de LaTeX %%
\DeclareMathOperator{\rel}{rel}
% colímit (Categories)
\DeclareMathOperator{\coker}{coker}
\DeclareMathOperator{\im}{im}
\DeclareMathOperator{\Tor}{Tor}

\title{\large{\textsc{\textbf{Teorema dels models acíclics}}}}
\author{\textsc{Jordi Cardiel}}
\date{}

\begin{document}
\maketitle
Denotarem per $\mathscr{T},\mathscr{K}$ les categories d'espais topològics i de complexos de cadenes (no-negatius). L'objectiu és generalitzar les idees de la demostració de l'axioma d'homotopia (per homologia singular) en un context categòric.\newline
Per demostrar l'axioma d'homotopia, és suficient veure que existeix una homotopia de cadenes $D=\{D_{n}^{X}:C_{n}(X)\rightarrow C_{n+1}(X\times\mathbb{I})\}$ entre $C_{\bullet}(\iota_{0}^{X}),C_{\bullet}(\iota_{1}^{X}):C_{\bullet}(X)\rightarrow C_{\bullet}(X\times\mathbb{I})$, on $\iota_{i}^{X}(x)=(x,i)$. L'aciclicitat de $H_{\bullet}(\Delta^{n}\times\mathbb{I})$, estendre per linealitat de $\mathscr{T}(\Delta^{n},X)$ a $C_{n}(X)$ i la naturalitat dels morfismes de $\mathscr{K}$ són els detalls tècnics cabdals per la construcció de l'homotopia de cadenes natural entre $C_{\bullet}(\iota_{0}^{X})$, $C_{\bullet}(\iota_{1}^{X})$. Donem una formulació més general.
\begin{definicio}
    Sigui $\mathscr{C}$ una categoria, $\mathscr{M}$ una col·lecció  d'objectes de $\mathscr{C}$ i $F:\mathscr{C}\rightarrow\mathscr{K}$ un functor covariant.
    \begin{enumerate}
        \item $F$ és acíclic en $\mathscr{M}$ si $F(M)$ és acíclic per tot $M\in\mathscr{M}$.
        \item $F$ és lliure en $\mathscr{M}$ si existeixen $M_{j}\in\mathscr{M}$ i $m_{j}\in F(M_{j})$ tals que $\mathscr{B}=\{F(f)(m_{j}):f\in\mathscr{C}(M_{j},X)\}$ és una base de $F(X)$ per tot $X\in\mathscr{C}$.
    \end{enumerate}
\end{definicio}
Direm que $\mathscr{C}$ és una categoria amb models $\mathscr{M}$ si $\mathscr{M}$ és una col·lecció d'objectes de $\mathscr{C}$. La condició d'aciclicitat ve motivada per l'aciclicitat de $H_{\bullet}(\Delta^{n}\times\mathbb{I})$; la condició de llibertat, per l'extensió per linealitat de $\mathscr{T}(\Delta^{n},X)$ a $C_{n}(X)$.
\begin{exemple}
    Sigui $\mathscr{C}=\mathscr{T}$, $\mathscr{M}=\{\Delta^{n}:n\geq0\}$ i $F=C_{\bullet}$. Com $\Delta^{n}$ són contràctils per tot $n\geq0$, $H_{\bullet}(\Delta^{n})$ és acíclic. Aleshores, $C_{\bullet}$ és acíclic en $\mathscr{M}$. Sigui $id_{\Delta^{n}}\in C_{\bullet}(\Delta^{n})$ la identitat. Tenim que
    \begin{align*}
        \mathscr{B}
        &=\{C_{\bullet}(f)(id_{\Delta^{n}}): f\in\mathscr{T}(\Delta^{n},X)\}\\
        &=\{f\circ id_{\Delta^{n}}:f\in\mathscr{T}(\Delta^{n},X)\}\\
        &=\{f:f\in\mathscr{T}(\Delta^{n},X)\}\\
        &=\mathscr{T}(\Delta^{n},X)
    \end{align*}
    Per tant, $\mathscr{B}$ són bases de $C_{\bullet}(X)$ i $C_{\bullet}$ és lliure en $\mathscr{M}$.
\end{exemple}
El resultat principal és el teorema dels models acíclics.
\begin{teorema}
    Sigui $\mathscr{C}$ una categoria amb models $\mathscr{M}$ i $F,G:\mathscr{C}\rightarrow\mathscr{K}$ functors covariants tals que $F$ és lliure en $\mathscr{M}$ i $G$ és acíclic en $\mathscr{M}$. Tota transformació natural $H_{0}(F)\rightarrow H_{0}(G)$ ve induïda per una única transformació natural $F\rightarrow G$ llevat d'una homotopia de cadenes (natural).
\end{teorema}
Veiem que l'axioma d'homotopia per homologia singular és conseqüència del teorema dels models acíclics.
\begin{corolari}[Axioma d'homotopia]
    Siguin $X,Y$ espais topològics, $f,g:X\rightarrow Y$ aplicacions contínues homòtopes. Aleshores, $f,g$ indueixen el mateix morfisme en homologia (singular).
\end{corolari}
\begin{proof}
    %És suficient veure que existeix una homotopia de cadenes $D=\{D_{n}^{X}:C_{n}(X)\rightarrow C_{n+1}(X\times\mathbb{I})\}$ entre $C_{\bullet}(\iota_{0}^{X}),C_{\bullet}(\iota_{1}^{X})$. 
    Considerem els functors covariants $C_{\bullet},G:\mathscr{T}\rightarrow\mathscr{K}$, $G$ definit per $G(X)=C_{\bullet}(X\times\mathbb{I})$ i $G(f)=f\times id_{\mathbb{I}}$. $C_{\bullet}$ és lliure en $\{\Delta^{n}:n\geq0\}$ i $G$ és acíclic en $\{\Delta^{n}\times\mathbb{I}:n\geq0\}$. $C_{\bullet}(\iota_{0}^{X}),C_{\bullet}(\iota_{1}^{X})$ indueixen la mateixa transformació natural $H_{0}(C_{\bullet})\rightarrow H_{0}(G)$. En efecte, si $\gamma:\mathbb{I}\rightarrow X\times\mathbb{I}$ és el camí definit per $\gamma(t)=(x,t)$, aleshores,
    \begin{align*}
        C_{0}(\iota_{0}^{X})(x)+\im{d_{1}^{X\times\mathbb{I}}}
        &=(x,0)+\im{d_{1}^{X\times\mathbb{I}}}
        &\quad&\textrm{(Per definició de $H_{0}(\iota_{0}^{X})$)}\\
        &=(x,0)+d_{1}^{X\times\mathbb{I}}\gamma+\im{d_{1}^{X\times\mathbb{I}}}
        &\quad&\textrm{($d_{1}^{X\times\mathbb{I}}\gamma\in\im{d_{1}^{X\times\mathbb{I}}}$)}\\
        &=(x,0)+(x,1)-(x,0)+\im{d_{1}^{X\times\mathbb{I}}}
        &\quad&\textrm{($d_{1}^{X\times\mathbb{I}}\gamma=(x,1)-(x,0)$)}\\
        &=(x,1)+\im{d_{1}^{X\times\mathbb{I}}}\\
        &=C_{0}(\iota_{1}^{X})(x)+\im{d_{1}^{X\times\mathbb{I}}}
        &\quad&\textrm{(Per definició de $H_{0}(\iota_{1}^{X})$)}
    \end{align*}
    La naturalitat és deu a la functorialitat de $H_{0}$ i que $f\times id_{\mathbb{I}}\circ\iota_{0}^{X}=\iota_{0}^{Y}\circ f$, $f\in\mathscr{T}(X,Y)$. Pel teorema dels models acíclics, existeix una homotopia de cadenes natural entre $C_{\bullet}(\iota_{0}^{X})$ i $C_{\bullet}(\iota_{1}^{X})$. Si existeix una homotopia de cadenes entre $C_{\bullet}(\iota_{0}^{X}),C_{\bullet}(\iota_{1}^{X})$ i $F:X\times\mathbb{I}\rightarrow Y$ és l'homotopia entre $f$ i $g$,
    \begin{align*}
        H_{\bullet}(f)
        &=H_{\bullet}(F\circ\iota_{0}^{X})
        &\quad&\textrm{($f=F\circ\iota_{0}^{X}$)}\\
        &=H_{\bullet}(F)\circ H_{\bullet}(\iota_{0}^{X})
        &\quad&\textrm{(Per functorialitat de $H_{\bullet}$)}\\
        &=H_{\bullet}(F)\circ H_{\bullet}(\iota_{1}^{X})
        &\quad&\textrm{($H_{\bullet}(\iota_{0}^{X})=H_{\bullet}(\iota_{1}^{X})$ ja que $C_{\bullet}(\iota_{0}^{X})\simeq C_{\bullet}(\iota_{1}^{X})$)}\\
        &=H_{\bullet}(F\circ\iota_{1}^{X})
        &\quad&\textrm{(Per functorialitat de $H_{\bullet}$)}\\
        &=H_{\bullet}(g)&\quad&\textrm{($g=F\circ\iota_{1}^{X}$)}
    \end{align*}
    que és el que volíem veure.
\end{proof}
Ara, veiem una conseqüència dels models acíclics que estableix una relació entre els grups d'homologia del producte d'espais topològics amb els grups d'homologia dels espais topològics.
\begin{corolari}[Eilenberg-Zilber]
    Sigui $X,Y$ espais topològics. Existeix una equivalència de cadenes natural entre $C_{\bullet}(X\times Y)$ i $C_{\bullet}(X)\otimes C_{\bullet}(Y)$   
\end{corolari}
\begin{proof}
    Considerem la categoria $\mathscr{T}\times\mathscr{T}$ de parells d'espais topològics i $\mathscr{M}=\{(\Delta^{n},\Delta^{m}):n,m\geq0\}$.\newline Considerem $F:\mathscr{T}\times\mathscr{T}\rightarrow\mathscr{K}$ el functor covariants definit per $F(X,Y)=C_{\bullet}(X,Y),F(f,g)=C_{\bullet}(f\times g)$. $F$ és acíclic i lliure en $\mathscr{M}$. En efecte, $H_{\bullet}(\Delta^{n}\times\Delta^{m})$ és acíclic ja que $\Delta^{n}\times\Delta^{m}$ és contràctil i, si $\delta_{n}\in C_{\bullet}(\Delta^{n}\times\Delta^{n})$ és el morfisme diagonal,
    \begin{align*}
        \mathscr{B}
        &=\{F(f,g)(\delta_{n}):(f,g)\in\mathscr{T}\times\mathscr{T}((\Delta^{n},\Delta^{n}),(X,Y))\}\\
        &=\{C_{\bullet}(f\times g)(\delta_{n}):(f,g)\in\mathscr{T}\times\mathscr{T}((\Delta^{n},\Delta^{n}),(X,Y))\}\\
        &=\{(f,g):(f,g)\in\mathscr{T}\times\mathscr{T}((\Delta^{n},\Delta^{n}),(X,Y))\}\\
        &=\mathscr{T}\times\mathscr{T}((\Delta^{n},\Delta^{n}),(X,Y))
    \end{align*}
    és base de $C_{\bullet}(X\times Y)$.\newline
    Considerem $G:\mathscr{T}\times\mathscr{T}\rightarrow\mathscr{K}$ el functor covariants definit per $G(X,Y)=C_{\bullet}(X)\otimes C_{\bullet}(Y),G(f,g)=C_{\bullet}(f)\otimes C_{\bullet}(g)$. $G$ és acíclic i lliure en $\mathscr{M}$. En efecte, per tot $q>0$, $H_{q}(C_{\bullet}(\Delta^{n})\otimes C_{\bullet}(\Delta^{m}))\cong\bigoplus_{r+s=q}H_{r}(\Delta^{n})\otimes H_{s}(\Delta^{m})\oplus\bigoplus_{r+s=q-1}\Tor(H_{r}(\Delta^{n}),H_{s}(\Delta^{m}))=0$ pel teorema algebraic de Künneth i perquè $H_{r}(\Delta^{n})$, $H_{s}(\Delta^{m})=0$ i
    \begin{align*}
        \mathscr{B}
        &=\{G(f,g)(id_{\Delta^{n}}\otimes id_{\Delta^{m}}):(f,g)\in\mathscr{T}\times\mathscr{T}((\Delta^{n},\Delta^{m}),(X,Y))\}\\
        &=\{(C_{\bullet}(f)\otimes C_{\bullet}(g))(id_{\Delta^{n}}\otimes id_{\Delta^{m}}):(f,g)\in\mathscr{T}\times\mathscr{T}((\Delta^{n},\Delta^{m}),(X,Y))\}\\
        &=\{f\otimes g:(f,g)\in\mathscr{T}\times\mathscr{T}((\Delta^{n},\Delta^{m}),(X,Y))\}\\
        &=\mathscr{T}(\Delta^{n},X)\otimes\mathscr{T}(\Delta^{m},Y)
    \end{align*}
    és base de $C_{\bullet}(X)\otimes C_{\bullet}(Y)$.\newline
    Siguin $X,Y\in\mathscr{T}$ i $\{X_{i},i\in\mathscr{I}\},\{Y_{j},j\in\mathscr{J}\}$ les famílies de components arc-connexes de $X,Y$, respectivament. Aleshores, $\{X_{i}\times Y_{j}:(i,j)\in\mathscr{I}\times\mathscr{J}\},\{X_{i}\otimes Y_{j}:(i,j)\in\mathscr{I}\times\mathscr{J}\}$ són bases de $H_{0}(X\times Y),H_{0}(C_{\bullet}(X)\otimes C_{\bullet}(Y))(\cong H_{0}(X)\otimes H_{0}(Y))$, respectivament. Definim les transformacions naturals $\rho_{-1}:H_{0}(X\times Y)\rightarrow H_{0}(C_{\bullet}(X)\otimes C_{\bullet}(Y)),\eta_{-1}:H_{0}(C_{\bullet}(X)\otimes C_{\bullet}(Y))\rightarrow H_{0}(X\times Y)$ com $\rho_{-1}(X_{i}\times Y_{j})=X_{i}\otimes Y_{j},\eta_{-1}(X_{i}\otimes Y_{j})=X_{i}\times Y_{j}$ i estenent per linealitat. Fixem-nos que $\eta_{-1}\circ\rho_{-1}=id_{H{0}(X\times Y)},\rho_{-1}\circ\eta_{-1}=id_{H_{0}(C_{\bullet}(X)\otimes C_{\bullet}(Y))}$.\newline
    Pel teorema dels models acíclics, existeixen transformacions naturals $\rho:C_{\bullet}(X\times Y)\rightarrow C_{\bullet}(X)\otimes C_{\bullet}(Y),\eta:C_{\bullet}(X)\otimes C_{\bullet}(Y)\rightarrow C_{\bullet}(X\times Y)$ que indueixen $\rho_{-1},\eta_{-1}$, respectivament. A més, $\eta\circ\rho$ i $id_{C_{\bullet}(X\times Y)}$ indueixen $id_{H_{0}(X\times Y)}$ (similarment, $\rho\circ\eta$ i $id_{C_{\bullet}(X)\otimes C_{\bullet}(Y)}$ indueixen $id_{H_{0}(C_{\bullet}(X)\otimes C_{\bullet}(Y))}$). Aleshores, pel teorema dels models acíclics, existeix una homotopia de cadenes natural entre $\eta\circ\rho$ i $id_{C_{\bullet}(X\times Y)}$ (similarment, existeix una homotopia de cadenes natural entre $\rho\circ\eta$ i $id_{C_{\bullet}(X)\otimes C_{\bullet}(Y)}$), com volíem veure.
\end{proof}
Ara, demostrem el teorema del models acíclics. La idea és la següent: construirem nivell a nivell la transformació natural $F\rightarrow G$ mitjançant inducció, com si a un gratacel fiquéssim pas a pas cada pis. El següent diagrama potser ajuda al enteniment del pas inductiu:
\begin{center}
    \begin{tikzcd}[row sep=scriptsize, column sep=scriptsize]
    &\vdots
    \arrow[dashed]{dd}{d_{n+1}'}
    &&\vdots
    \arrow[dashed]{dd}{\Delta_{n+1}'}\\
    \vdots
    \arrow{dd}{d_{n+1}}
    &&\vdots
    \arrow{dd}[near start]{\,\Delta_{n+1}}&\\
    %% Dimensió n %%
    &F_{n}(M_{j})
    \arrow[dashed]{dl}{F_{n}(f)}
    \arrow[dashed, near start]{rr}{\phi_{n}(M_{j})} 
    \arrow[dashed]{dd}[near end]{d_{n}'}
    &&G_{n}(M_{j})
    \arrow[dashed]{dl}{G_{n}(f)}
    \arrow[dashed]{dd}{\Delta_{n}'}\\
    F_{n}(X)
    \arrow[crossing over]{rr}[near end]{\phi_{n}(X)}
    \arrow{dd}{d_{n}}
    &&G_{n}(X)
    \arrow{dd}[near start]{\,\Delta_{n}}
    \arrow[from=uu,crossing over]\\
    %% Dimensió n-1 %%
    &F_{n-1}(M_{j})
    \arrow[dashed]{dl}{F_{n-1}(f)}
    \arrow[dashed, near start]{rr}{\phi_{n-1}(M_{j})} 
    \arrow[dashed]{dd}[near end]{d_{n-1}'}
    &&G_{n-1}(M_{j})
    \arrow[dashed]{dl}{G_{n-1}(f)}
    \arrow[dashed]{dd}{\Delta_{n-1}'}\\
    F_{n-1}(X)
    \arrow[crossing over]{rr}[near end]{\phi_{n-1}(X)}
    \arrow{dd}{d_{n-1}}
    &&G_{n-1}(X)
    \arrow{dd}[near start]{\,\Delta_{n-1}}
    \arrow[from=uu,crossing over]\\
    %% Dimensió n-2 %%
    &F_{n-2}(M_{j})
    \arrow[dashed]{dl}{F_{n-2}(f)}
    \arrow[dashed, near start]{rr}{\phi_{n-2}(M_{j})} 
    \arrow[dashed]{dd}[near end]{d_{n-2}'}
    &&G_{n-2}(M_{j})
    \arrow[dashed]{dl}{G_{n-2}(f)}
    \arrow[dashed]{dd}{\Delta_{n-2}'}\\
    F_{n-2}(X)
    \arrow[crossing over]{rr}[near end]{\phi_{n-2}(X)}
    \arrow{dd}{d_{n-2}}
    &&G_{n-2}(X)
    \arrow{dd}[near start]{\,\Delta_{n-2}}
    \arrow[from=uu,crossing over]\\
    &\vdots
    \arrow[dashed]{dd}{d_{1}'}
    &&\vdots
    \arrow[dashed]{dd}{\Delta_{1}'}\\
    \vdots
    \arrow{dd}{d_{1}}
    &&\vdots
    \arrow{dd}[near start]{\,\Delta_{1}}&\\
    &F_{0}(M_{j})
    \arrow[dashed]{dl}{F_{0}(f)}
    \arrow[dashed, near start]{rr}{\phi_{0}(M_{j})} 
    \arrow[dashed]{dd}[near end]{\varepsilon_{F(M_{j})}}
    &&G_{0}(M_{j})
    \arrow[dashed]{dl}{G_{0}(f)}
    \arrow[dashed]{dd}{\varepsilon_{G(M_{j})}}\\
    F_{0}(X)
    \arrow[crossing over]{rr}[near end]{\phi_{0}(X)}
    \arrow{dd}{\varepsilon_{F(X)}}
    &&G_{0}(X)
    \arrow{dd}[near start]{\,\varepsilon_{G(X)}}
    \arrow[from=uu,crossing over]
    \\
    &H_{0}(F(M_{j}))
    \arrow[dashed]{dl}{H_{0}(F)(f)}
    \arrow[dashed, swap]{rr}[near start]{\Phi(M_{j})}
    &&H_{0}(G(M_{j}))
    \arrow[dashed]{dl}{H_{0}(G)(f)}\\
    H_{0}(F(X))
    \arrow[swap]{rr}{\Phi(X)}
    &&H_{0}(G(X))
    \arrow[from=uu,crossing over]
    \end{tikzcd}
\end{center}
\begin{proof}[Demostració (Models acíclics).]
    Sigui $\Phi:H_{0}(F)\rightarrow H_{0}(G)$ una transformació natural. Com $F$ és lliure en $\mathscr{M}$, existeixen $M_{j}\in\mathscr{M}$ i $m_{j}\in F(M_{j})$ tals que $\mathscr{B}_{F(X)}$ és una base de $F(X)$. Sigui $m_{j}\in\mathscr{B}_{F(X)}\cap F_{0}(M_{j})$. Com $G$ és acíclic en $\mathscr{M}$, la successió $G_{0}(M_{j})\rightarrow H_{0}(G(M_{j}))\rightarrow0$ és exacta. Per exactitud, $\varepsilon_{G(M_{j})}:G_{0}(M_{j})\rightarrow H_{0}(G(M_{j}))$ és exhaustiva. Com $\Phi(M_{j})\varepsilon_{F(M_{j})}(m_{j})\in H_{0}(G(M_{j}))$, per exhaustivitat de $\varepsilon_{G(M_{j})}$,
    \begin{equation*}
        \exists m_{j}'(m_{j}'\in G_{0}(M_{j})\land\varepsilon_{G(M_{j})}(m_{j}')=\Phi(M_{j})\varepsilon_{F(M_{j})}(m_{j}))
    \end{equation*}
    %Considerem $\phi_{0}(M_{j}):F_{0}(M_{j})\rightarrow G_{0}(M_{j})$ definida per $\phi_{0}(M_{j})(m_{j})=m_{j}'$. Per construcció,
    %\begin{align*}
    %    \varepsilon_{G(M_{j})}\phi_{0}(M_{j})(m_{j})
    %    &=\varepsilon_{G(M_{j})}(m_{j}')
    %    &\quad&\textrm{(Per definició de $\phi_{0}(M_{j})$)}\\
    %    &=\Phi(M_{j})\varepsilon_{F(M_{j})}(b_{j})
    %    &\quad&\textrm{(Per definició de $m_{j}'$)}
    %\end{align*}
    Sigui $X$ un objecte de $\mathscr{C}$ i $f\in\mathscr{C}(M_{j},X)$. Definim $\phi_{0}(X):F_{0}(X)\rightarrow G_{0}(X)$ sobre $\mathscr{B}_{F_{0}(X)}$ com\newline
    $\phi_{0}(X)(F_{0}(f)(m_{j}))=G_{0}(f)(m_{j}')$ i estenent a $F(X)$. Veiem que $\varepsilon_{G(X)}\phi_{0}(X)=\Phi(X)\varepsilon_{F(X)}$. En efecte,
    \begin{align*}
        \varepsilon_{G(X)}\phi_{0}(X)F_{0}(f)(m_{j})
        &=\varepsilon_{G(X)}G_{0}(f)(m_{j}')
        &\quad&\textrm{(Per definició de $\phi_{0}(X)$)}\\
        &=H_{0}(G)(f)\varepsilon_{G(M_{j})}(m_{j}')
        &\quad&\textrm{(Per naturalitat dels morfismes de $\mathscr{K}$)}\\
        &=H_{0}(G)(f)\Phi(M_{j})\varepsilon_{F(M_{j})}(m_{j})
        &\quad&\textrm{(Per definició de $m_{j}'$)}\\
        &=\Phi(X)H_{0}(F)(f)\varepsilon_{F(M_{j})}(m_{j})
        &\quad&\textrm{(Per naturalitat de $\Phi$)}\\
        &=\Phi(X)\varepsilon_{F(X)}F_{0}(f)(m_{j})
        &\quad&\textrm{(Per naturalitat dels morfismes de $\mathscr{K}$)}
    \end{align*}
    \begin{center}
    \begin{tikzcd}[row sep=scriptsize, column sep=scriptsize]
    &F_{0}(M_{j})
    \arrow[dashed]{dl}{F_{0}(f)}
    %\arrow[dashed, near start]{rr}{\phi_{0}(M_{j})} 
    \arrow[dashed]{dd}[near end]{\varepsilon_{F(M_{j})}}
    &&G_{0}(M_{j})
    \arrow[dashed]{dl}{G_{0}(f)}
    \arrow[dashed]{dd}{\varepsilon_{G(M_{j})}}\\
    F_{0}(X)
    \arrow[crossing over]{rr}[near end]{\phi_{0}(X)}
    \arrow{dd}{\varepsilon_{F(X)}}
    &&G_{0}(X)
    \arrow{dd}[near start]{\,\varepsilon_{G(X)}}
    \\
    &H_{0}(F(M_{j}))
    \arrow[dashed]{dl}{H_{0}(F)(f)}
    \arrow[dashed, swap]{rr}[near start]{\Phi(M_{j})}
    &&H_{0}(G(M_{j}))
    \arrow[dashed]{dl}{H_{0}(G)(f)}\\
    H_{0}(F(X))
    \arrow[swap]{rr}{\Phi(X)}
    &&H_{0}(G(X))
    \arrow[from=uu,crossing over]
    \end{tikzcd}
    \end{center}
    Sigui $g\in\mathscr{C}(X,Y)$. Veiem que $\phi_{0}$ és natural. En efecte,
    \begin{align*}
        G_{0}(g)\phi_{0}(X)F_{0}(f)(m_{j})
        &=G_{0}(g)G_{0}(f)(m_{j}')
        &\quad&\textrm{(Per definició de $\phi_{0}(X)$)}\\
        &=G_{0}(g\circ f)(m_{j}')
        &\quad&\textrm{(Per functorialitat de $G_{0}$)}\\
        &=\phi_{0}(Y)F_{0}(g\circ f)(m_{j})
        &\quad&\textrm{(Per definició de $\phi_{0}(Y)$ i perquè $g\circ f\in\mathscr{C}(M_{j},Y)$)}\\
        &=\phi_{0}(Y)F_{0}(g)F_{0}(f)
        &\quad&\textrm{(Per functorialitat de $F_{0}$)}
    \end{align*}
    Suposem que $\phi_{0},\ldots,\phi_{n-1}$ són transformacions naturals tals que $\Delta_{k}\phi_{k}(X)=\phi_{k-1}(X)d_{k}$ per $0<k<n$ i $\varepsilon_{G(X)}\phi_{0}(X)=\Phi(X)\varepsilon_{F(X)}$. Sigui $m_{j}\in\mathscr{B}_{F(X)}\cap F_{n}(M_{j})$. Tenim que
    \begin{align*}
        \Delta_{n-1}'\phi_{n-1}(M_{j})d_{n}'(m_{j})
        &=\phi_{n-2}(M_{j})d_{n-1}'d_{n}'(m_{j})
        &\quad&\textrm{(Per hipòtesi d'inducció)}\\
        &=\phi_{n-2}(M_{j})(0)=0
        &\quad&\textrm{($d_{n-1}'d_{n}'=0$)}
    \end{align*}
    Aleshores, $\phi_{n-1}(M_{j})d_{n}'(m_{j})\in\ker\Delta_{n-1}'$. Com $G$ és acíclic en $\mathscr{M}$, $\im\Delta_{n}'=\ker\Delta_{n-1}'$. Per tant,
    \begin{equation*}
        \exists m_{j}'(m_{j}'\in G_{n}(M_{j})\land\Delta_{n}'(m_{j}')=\phi_{n-1}(M_{j})d_{n}'(m_{j}))
    \end{equation*}
    Definim $\phi_{n}(X):F_{n}(X)\rightarrow G_{n}(X)$ sobre $\mathscr{B}_{F_{n}(X)}$ com $\phi_{n}(X)(F_{n}(f)(m_{j}))=G_{n}(f)(m_{j}')$ i estenent a $F_{n}(X)$. Veiem que $\Delta_{k}\phi_{k}(X)=\phi_{k-1}(X)d_{k}$. En efecte,
    \begin{align*}
        \Delta_{n}\phi_{n}(X)F_{n}(f)(m_{j})
        &=\Delta_{n}G_{n}(f)(m_{j}')
        &\quad&\textrm{(Per definició de $\phi_{n}(X)$)}\\
        &=G_{n-1}(f)\Delta_{n}'(m_{j}')
        &\quad&\textrm{(Per naturalitat dels morfismes de $\mathscr{K}$)}\\
        &=G_{n-1}(f)\phi_{n-1}(M_{j})d_{n}'(m_{j})&\quad&\textrm{(Per definició de $m_{j}'$)}\\
        &=\phi_{n-1}(X)F_{n-1}(f)d_{n}'(m_{j})&\quad&\textrm{(Per naturalitat $\phi_{n-1}$)}\\
        &=\phi_{n-1}(X)d_{n}F_{n}(f)(m_{j})&\quad&\textrm{(Per naturalitat dels morfismes de $\mathscr{K}$)}
    \end{align*}
    \begin{center}
    \begin{tikzcd}[row sep=scriptsize, column sep=scriptsize]
    %% Dimensió n %%
    &F_{n}(M_{j})
    \arrow[dashed]{dl}{F_{n}(f)}
    %\arrow[dashed, near start]{rr}{\phi_{n}(M_{j})} 
    \arrow[dashed]{dd}[near end]{d_{n}'}
    &&G_{n}(M_{j})
    \arrow[dashed]{dl}{G_{n}(f)}
    \arrow[dashed]{dd}{\Delta_{n}'}\\
    F_{n}(X)
    \arrow[crossing over]{rr}[near end]{\phi_{n}(X)}
    \arrow{dd}{d_{n}}
    &&G_{n}(X)
    \arrow{dd}[near start]{\,\Delta_{n}}\\
    %% Dimensió n-1 %%
    &F_{n-1}(M_{j})
    \arrow[dashed]{dl}{F_{n-1}(f)}
    \arrow[dashed, near start, swap]{rr}{\phi_{n-1}(M_{j})}
    &&G_{n-1}(M_{j})
    \arrow[dashed]{dl}{G_{n-1}(f)}\\
    F_{n-1}(X)
    \arrow[crossing over, swap]{rr}{\phi_{n-1}(X)}
    &&G_{n-1}(X)
    \arrow[from=uu,crossing over]
    \end{tikzcd}
    \end{center}
    La naturalitat de $\phi_{n}$ es comprova igual que la naturalitat de $\phi_{0}$.\newline
    Sigui $\psi:F\rightarrow G$ una transformació natural que indueix $\Phi$. Definim una homotopia de cadenes natural inductivament. Sigui $m_{j}\in F_{0}(M_{j})$. Tenim que
    \begin{align*}
        &\varepsilon_{G(M_{j})}((\phi_{0}(M_{j})-\psi_{0}(M_{j}))(m_{j}))\\
        =\,&\varepsilon_{G(M_{j})}(\phi_{0}(M_{j})(m_{j}))
        -\varepsilon_{G(M_{j})}(\psi_{0}(M_{j})(m_{j}))\\
        =\,&\Phi(M_{j})\varepsilon_{F(M_{j})}(m_{j})
        -\Phi(M_{j})\varepsilon_{F(M_{j})}(m_{j})\\
        =\,&0
        &\quad&\textrm{($\phi,\psi$ indueixen $\Phi$)}
    \end{align*}
    Com $G$ és acíclic en $\mathscr{M}$, $\ker{\varepsilon_{G(M_{j})}}=\im{\Delta_{1}'}$, d'on $\exists m_{j}'(m_{j}'\in G_{1}(M_{j})\land\Delta_{1}'m_{j}'=(\phi_{0}(M_{j})-\psi_{0}(M_{j}))(m_{j}))$. Com $F$ és lliure en $\mathscr{M}$, definim $D_{0}^{X}:F_{0}(X)\rightarrow G_{1}(X)$ per $D_{0}^{X}(F_{0}(f)(m_{j}))=_{0}G(f)(m_{j}')$ i estenent per linealitat. Tenim que $\Delta_{1}D_{0}^{X}=\phi_{0}(X)-\psi_{0}(X)$.\newline
    Suposem que $D_{0}^{X},\ldots,D_{n-1}^{X}$ són tals que $\Delta_{k+1}D_{k}^{X}+D_{k-1}^{X}d_{k}=\phi_{k}(X)-\psi_{k}(X)$ per $0<k<n$ i $\Delta_{1}D_{0}^{X}=\phi_{0}(X)-\psi_{0}(X)$. Sigui $m_{j}\in F_{n}(M_{j})$. Tenim que
    \begin{align*}
        &\Delta_{n}'(\phi_{n}(M_{j})-\psi_{n}(M_{j})-D_{n-1}^{M_{j}}d_{n}')(m_{j})\\
        =\,&(\Delta_{n}'\phi_{n}(M_{j})
        -\Delta_{n}'\psi_{n}(M_{j})
        -\Delta_{n}'D_{n-1}^{M_{j}}d_{n}')(m_{j})\\
        =\,&(\phi_{n-1}(M_{j})d_{n}'
        -\psi_{n-1}(M_{j})d_{n}'
        -(\phi_{n-1}(M_{j})
        -\psi_{n-1}(M_{j})
        -D_{n-2}^{M_{j}}d_{n-1}')d_{n}')(m_{j})
        &\quad&\textrm{(Per inducció)}\\
        =\,&(\phi_{n-1}(M_{j})d_{n}'
        -\psi_{n-1}(M_{j})d_{n}'
        -\phi_{n-1}(M_{j})d_{n}'
        +\psi_{n-1}(M_{j})d_{n}'
        +D_{n-2}^{M_{j}}d_{n-1}'d_{n}')(m_{j})
        &\quad&\textrm{($d_{n-1}'d_{n}'=0$)}\\
        =\,&0
    \end{align*}
    Com $G$ és acíclic en $\mathscr{M}$, $\ker{\Delta_{n}'}=\im{\Delta_{n+1}'}$, d'on 
    \begin{equation*}
        \exists m_{j}'(m_{j}'\in G_{n+1}(M_{j})\land\Delta_{n+1}'m_{j}'=(\phi_{n}(M_{j})-\psi_{n}(M_{j})-D_{n-1}^{M_{j}}d_{n}')(m_{j}))
    \end{equation*}
    Com $F$ és lliure en $\mathscr{M}$, definim $D_{n}^{X}:F_{n}(X)\rightarrow G_{n+1}(X)$ per $D_{n}^{X}(F_{n}(f)(m_{j}))=G_{n}(f)(m_{j}')$ i estenent per linealitat. Per definició, tenim que $\Delta_{n+1}D_{n}^{X}+D_{n-1}^{X}d_{n}=\phi_{n}(X)-\psi_{n}(X)$. 
\end{proof}
\end{document}